%chktex-file 46
%chktex-file 9
%chktex-file 10

Concrete is a most fundamental material for buildings, bridges, tunnels and other structures with safety-critical containment functions like in nuclear power plants and radioactive waste disposals.
This is because of its proven strength, durability and cost-effectiveness.
However, environmental exposure can accelerate its degradation~\cite{GLASSER2008226}. For example, structures located in coastal or urban areas, where humidity, chlorides, and fluctuating temperatures accelerate deterioration, are especially prone to corrosion-related process~\cite{qu2021durability}.
One such process is carbonation, where atmospheric carbon dioxide ($CO_2$) reacts with calcium compounds by forming calcium carbonate, challenging the reinforced concrete structures through the reduction of the alkalinity of the concrete, that is, lowering the PH, which compromises the protective layer surrounding the embedded steel reinforcement.
As carbonation progresses and pH decreases, this protective layer breaks down, leaving steel vulnerable to corrosion~\cite{carbonation}.
This degradation can lead to cracks, cracking, and weakening of the bond between the steel and concrete, ultimately diminishing the structural durability.
Another degradation mechanism is chloride-induced corrosion, which leads to the accumulation of chloride ions around the embedded steel reinforcement. Unlike carbonation, chloride ions can penetrate even high-alkalinity concrete without necessarily lowering the pH, directly compromising the passive film on steel.
Once a critical chloride concentration threshold is reached on the steel surface, localized corrosion begins, often in the form of pitting~\cite{shi2012durability}.
The process of corrosion of concrete structures must be taken into account for safety-critical applications on long-term periods, like for the containment barriers of radioactive waste repositories~\cite{walton1990models}.
To maintain structural integrity over such long service lifes, the risk assessment must consider the environmental stress uncertainties, as they evolve over the long time horizons~\cite{lindborg2018climate}.
\\

Traditional risk assessment tools, such as Event Trees (ET)~\cite{papazoglou_mathematical_1998} and Fault Trees (FT)~\cite{kabir_overview_2017}, are the most widely used for the risk assessment of nuclear power plants and radioactive waste disposals.
However, they are inherently limited by Boolean logic and the treatment of dependencies.
Bayesian Networks (BNs), which model system dependencies through directed acyclic graphs~\cite{mahadevan2001bayesian} can also account for the uncertainties due to imperfect knowledge on system behaviour, include common cause failures, and consider multistate events. \\
A BN enables diagnostic problem-solving while preserving the multi-scenario analysis and minimal cut-set identification features inherent to FT analysis.
Since risk assessment often involves conditional probabilities, BNs provide an efficient factorization of joint probability distributions, making them particularly useful for reliability analysis, as shown by the work of~\textcite{langseth_bayesian_2007}.
BNs have been used in several applications, including maintenance planning of structural components~\cite{morato2022optimal}, multi-hazard fragility assessment of bridges~\cite{gehl2016development, barros2024gaussian}, disaggregation of structural failure~\cite{yazdani2020bayesian}, deterioration processes in engineering structures~\cite{luque2019risk,lee2023dynamic,tran2020dynamic}, and especially the reliability assessment of reinforced concrete structures~\cite{hackl2016reliability,hosseini2024dynamic,guo2024mixed}.\\
Despite their advantages, conventional BNs are limited in their ability to model aleatoric uncertainties, as they primarily rely on expert knowledge and discrete probability assignments.
Given that the uncertainties in corrosion dynamics are strongly influenced by $CO_2$ concentration, temperature, and relative humidity, it is essential to incorporate climate change projections.\\
In this paper, we propose an enhanced Bayesian Network (eBN) with the incorporation of imprecise probabilities for handling both aleatoric and epistemic uncertainties.
The eBN framework integrates structural reliability methods, such as First-Order Reliability Methods (FORM), and advanced Monte Carlo techniques into conventional BN structures. 
The extension allows eBNs to: 
\begin{enumerate*}[label=\roman*)]
  \item incorporate aleatoric uncertainties by introducing continuous nodes associated with probability distributions, significantly enhancing their applicability to risk assessment;
  \item to combine the probabilistic reasoning capabilities with well-established reliability methods;
  \item to adopt imprecise probabilities into the eBN framework, incorporating both epistemic and aleatoric uncertainties into the assessment process;
  \item to model complex interactions between environmental variables and material properties over time.
\end{enumerate*} 
Thus, the eBN framework enables the embedding of climate change projections variables, e.g., $CO_2$ concentrations, temperature and humidity, into the corrosion process and allows a scenario-based evaluation of the impact of the environmental conditions over the chemical degradation of concrete in time.
With the ability to assess climate change scenarios, eBNs offer an innovative framework, enabling the assessment of the resilience of concrete structures in the face of evolving environmental stressors, ultimately enabling risk-informed decision-making for adaptation and mitigation strategies.\\

The eBN modelling framework is applied to the near-surface radioactive waste disposal located in Dessel, Belgium~\cite{tosoni_comprehensiveness_2019-1}.
Three climate change scenarios are considered, each corresponding to a distinct Shared Socioeconomic Pathway (\textit{ssp}) relevant to the region~\cite{CMIP6}.
These projections are used to extrapolate the temporal evolution that influences the distribution of relevant parameters involved in carbonation-induced and chloride-induced corrosion processes.
The primary objectives of this study are to: 
\begin{enumerate*}[label=\roman*)]
    \item  develop an eBN modelling framework tailored to the Dessel disposal that accounts for aleatoric uncertainties affecting structural integrity;
    \item quantify the impact of epistemic uncertainty associated with climate change projections; and
    \item support informed decision-making through a multi-scenario risk assessment.
\end{enumerate*}
\\

The remaining sections of the manuscript are structured as follows: Section~\ref{ebn} formulates the eBN-based framework with incorporation of imprecise probabilities; Section~\ref{caseofstudy} describes the case study regarding two relevant degradation processes, carbonation-induced and chloride-induced corrosions, affecting concerete structures related to radioactive waste disposals; and Section~\ref{results} presents the results. In Section~\ref{conclusions}, conclusions are drawn.

% Risk assessment plays a pivotal role in ensuring the safety and reliability of complex engineering systems, particularly in safety-critical domains such as aerospace, nuclear power, and offshore industries.
% The increasing complexity of these systems necessitates robust methodologies to prevent design and operational failures, optimize safety criteria, and avoid excessive conservatism in risk evaluations.
% A well-structured risk assessment process is essential for guiding decision-making and preventing catastrophic events, as illustrated by past disasters such as the Piper Alpha offshore oil explosion~\cite{Piper_alpha}, the Seveso industrial accident~\cite{Seveso}, and the Clapham Junction rail crash!\cite{Clapham}. \\
% To be effective, a risk assessment framework must be both accurate and comprehensive, capable of accounting for multiple failure scenarios, uncertainties, and the varying quality of available information. Moreover, the framework should support decision-making in a computationally efficient manner, balancing precision with feasibility. \\
% Traditional risk assessment tools, such as Event Trees (ET)~\cite{papazoglou_mathematical_1998} and Fault Trees (FT)~\cite{kabir_overview_2017}, are the most widely used for prognosis analysis but are inherently limited by Boolean logic, making them unsuitable for an exhaustive dependability analysis. Bayesian Networks (BNs), which model system dependencies through directed acyclic graphs, can be seen as generalized FTs. BNs framework can model uncertainties in the behavior of the \textit{and} and \textit{or} gates, express imperfect knowledge on system behavior by incorporating Common Cause Failures and consider multi-state events, by including probabilistic dependencies. Furthermore, a BN enables diagnostic problem-solving while preserving the multi-scenario analysis and minimal cut-set identification features inherent to FT analysis.
% Since risk assessment often involves conditional probabilities, BNs provide an efficient factorization of joint probability distributions, making them particularly useful for reliability analysis as shown by the work of~\textcite{langseth_bayesian_2007}. \\
% Despite their advantages, traditional BNs are limited in their ability to model aleatoric uncertainties, as they primarily rely on expert knowledge and discrete probability assignments.\\

% \subsection{Concrete structures}
% Concrete is one of the most fundamental materials in modern infrastructure, forming the backbone of buildings, bridges, tunnels, and various critical structures.
% Its widespread use is due to its strength, durability, and cost-effectiveness.
% Beyond its mechanical properties, concrete interacts with the environment over time, undergoing chemical processes that influence both its performance and sustainability. 
% One such process is carbonation, where atmospheric carbon dioxide ($CO₂$) reacts with calcium compounds in hardened concrete, forming calcium carbonate.
% This natural reaction allows concrete to absorb CO₂ throughout its lifespan, contributing to carbon sequestration and reducing its overall environmental footprint. While carbonation provides potential environmental benefits, it also presents challenges for reinforced concrete structures. One of the primary concerns is the reduction of concrete’s alkalinity, which compromises the protective layer surrounding embedded steel reinforcement. In fresh concrete, the highly alkaline environment naturally forms a passive film on the steel, preventing it from corroding.
% However, as carbonation progresses and the pH decreases, this protective layer breaks down, leaving the steel vulnerable to corrosion~\cite{carbonation}.
% This deterioration can lead to cracks, spalling, and weakening of the bond between the steel and concrete, ultimately diminishing the structural capacity of the system. Given the widespread use of reinforced concrete in critical infrastructure, it is essential to account for corrosion risks to ensure long-term reliability and safety. \\
% Assessing the risk of corrosion in concrete structures is particularly important in safety-sensitive applications such as radiocative waste repositories. The extent of degradation depends on factors such as environmental exposure, material composition, and design considerations. Structures located in coastal or urban areas, where humidity, chlorides, and fluctuating temperatures accelerate deterioration, are especially prone to corrosion-related failures. To maintain structural integrity and extend service life, risk assessment frameworks must incorporate these environmental and material factors, allowing for the development of predictive models and preventive maintenance strategies. \\
% Failing to account for corrosion in the risk assessment of concrete structures can lead to severe safety risks and significant financial consequences. Historical failures, such as bridge collapses and the deterioration of aging nuclear containment facilities, emphasize the necessity of evaluating long-term structural durability. However, traditional risk assessment methods often rely on current environmental conditions without fully considering the evolving impact of climate change. Given that carbonation and corrosion rates are strongly influenced by CO₂ concentration, temperature, and relative humidity, it is essential to incorporate climate change projections into risk analysis tools to ensure more accurate assessments of structural longevity.
% Rising atmospheric CO₂ levels can accelerate the carbonation process, reducing the alkalinity of concrete at a faster rate and compromising the passive protection of embedded steel. Similarly, increasing global temperatures can enhance the diffusion of carbon dioxide into concrete, further amplifying carbonation effects. At the same time, shifts in relative humidity play a critical role in corrosion dynamics. Higher humidity levels promote corrosion once carbonation has reached the reinforcement, while excessively dry conditions can slow carbonation. These interdependent factors suggest that static, present-day environmental assumptions may underestimate long-term structural degradation risks. \\

% \subsection{Proposed solution}
% To address this, it is crucial to integrate probabilistic modeling approaches that incorporate climate projections, allowing for scenario-based evaluations of how changing environmental conditions will impact the deterioration of reinforced concrete over time. By employing models that account for uncertainty in climate variables, engineers can develop more adaptive and forward-looking maintenance strategies, ensuring the resilience of infrastructure in the face of evolving environmental stressors. 
% A comprehensive risk assessment framework must therefore move beyond conventional methodologies and embrace an uncertainty-aware approach that integrates climate science, probabilistic modeling, and structural reliability analysis. \\
% The proposed solution to address the limitations of traditional BNs regarding aletoric uncertainties and the need of tool capable to deal with epistemic uncertainties is the Enhanced Bayesian Network (eBN) with imprecise probabilities.
% The eBN framework integrates structural reliability methods—such as Monte Carlo simulation, First-Order Reliability Methods (FORM), and Advanced Monte Carlo techniques—into traditional BN structures. This extension allows eBNs to incorporate aleatoric uncertainties by introducing continuous nodes associated with probability distributions, significantly enhancing their applicability to risk assessment. By combining the probabilistic reasoning capabilities of BNs with well-established reliability methods, eBNs provide a powerful and flexible tool for assessing risks in engineering systems, ultimately improving decision-making processes and system safety.\\
% Introduction of imprecise probabilities into the eBN framework addresses the problem of a reliable risk assessment. By incorporating both epistemic uncertainties into the assessment process. The eBNs with imprecise probabilities provide a robust and flexible framework capable of modeling the complex interactions between environmental variables and material properties over time. This approach enables the incorporation of climate projections—such as CO₂ concentrations, temperature, and humidity—into risk models, facilitating a more comprehensive understanding of how these factors influence the long-term durability of concrete structures. With the ability to assess multiple failure scenarios under changing environmental conditions, eBNs offer a forward-thinking tool that enhances the predictive power and reliability of risk assessments, ultimately enabling better-informed decision-making regarding maintenance, repair, and mitigation strategies.
