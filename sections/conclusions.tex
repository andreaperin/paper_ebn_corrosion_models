Traditional performance assessment of radioactive waste disposals has relied mainly on deterministic framework, based on conservative assumptions. These do not explicitly account for the many uncertainties that affect the assessment. 
In this work, we have developed a framework of imprecise eBN to model concrete corrosion considering epistemic and aleatoric uncertainties, including those stemming from climatic change projections, with a focus on concrete integrity as the primary factor influencing the structural reliability of long-term containment.\\
The adoption of imprecise eBNs enables the aggregation of scenario results within a framework that supports exact-inference algorithms to perform prognostic analysis, for informing decision-making for risk management. Through the use of imprecise probabilities, the framework allows explicitly addressing epistemic uncertainty due to sparse or incomplete data.\\
Although this study has focused on a specific yet critical aspect of radioactive waste disposal performance assessment, namely, the chemical degradation of concrete, the proposed framework is inherently modular and extensible. It can readily accommodate additional degradation mechanisms or coupled physical processes, enabling a more comprehensive safety assessment. Ultimately, such a generalized and uncertainty-aware approach has the potential to complement traditional deterministic frameworks in support of licensing decisions for radioactive waste repositories.\\
The concrete chemical degradation has been described using two models for carbonation-induced and chloride-induced corrosion, showing coherent results in that the failure probabilities consistently fall within the bounds defined by the imprecise intervals.

Despite its advantages, the eBN framework presents certain limitations on the computational side. In fact, precise eBN relies on Monte Carlo simulation for functional node evaluation, which leads to high computational costs. This computational burden is even more exacerbated in the imprecise case due to the need for nested simulations and global optimization, as seen in the Double Loop Monte Carlo and Random Slicing methods.
Future developments should consider incorporating additional degradation phenomena to enrich the risk assessment scope. A unified modelling node for the combined effects of carbonation and chloride-induced corrosion could also be considered, avoiding oversimplified strategies such as logical \textit{OR-gate} or additive approaches, which respectively under- or overestimate the risks. Moreover, optimization of control nodes for parameters like chloride concentration and relative humidity should be undertaken with respect to operational costs and performance. Finally, alternative approximate inference strategies, such as Non Intrusive Stochastic Simulation (NISS)~\cite{wei_non-intrusive_2019} and Collaborative and Adaptive Bayesian Optimization (CABO)~\cite{hong_sequential_2024}, merit investigation to alleviate the computational demand of the imprecise eBN framework.