In this work, we developed a framework of imprecise eBN to model concrete corrosion under epistemic and aleatoric uncertainties, primarily stemming from climatic change projections. 
Specifically, we considered the safety-critical context of radioactive waste disposals, with a focus on concrete integrity as the primary factor influencing the structural reliability of long-term containment. 
The process of concrete chemical degradation has been described using two models for carbonation-induced and chloride-induced corrosion, showing coherent results in that the failure probabilities consistently fall within the bounds defined by the imprecise intervals.

The adoption of imprecise eBNs enabled aggregation of scenario-based results within a framework that supports exact-inference algorithms to perform prognostic analysis, facilitating and informing decision-making for risk management. Furthermore, the framework allows addressing epistemic uncertainties due to sparse or incomplete data through the use of imprecise probabilities.

Despite its advantages, the eBN framework presents certain limitations on the computational side. In fact, precise eBN relies on Monte Carlo simulation for functional node evaluation, which leads to high computational costs. This computational burden is even more exacerbated in the imprecise case due to the need for nested simulations and global optimization, as seen in the Double Loop Monte Carlo and Random Slicing methods.

Future developments should consider incorporating additional degradation phenomena to enrich the risk assessment scope. A unified modelling node for the combined effects of carbonation and chloride-induced corrosion could also be considered, avoiding oversimplified strategies such as logical \textit{OR-gate} or additive approaches, which respectively under- or overestimate the risks. Moreover, optimization of control nodes for parameters like chloride concentration and relative humidity should be undertaken with respect to operational costs and performance. Finally, alternative approximate inference strategies, such as Non Intrusive Stochastic Simulation (NISS)~\cite{wei_non-intrusive_2019} and Collaborative and Adaptive Bayesian Optimization (CABO)~\cite{hong_sequential_2024}, merit investigation to alleviate the computational demand of the imprecise eBN framework.