%chktex-file 46
The susceptibility of concrete structures to chemical corrosion is significantly influenced by environmental exposure, a condition intensified by climate change. 
Bayesian Networks have demonstrated significant effectiveness in modeling the interdependencies inherent in corrosion processes. However, traditional Bayesian Networks encounter challenges when addressing aleatoric and epistemic uncertainties and accommodating the evolving impacts of climate change on corrosion dynamics over time. In response, we propose an advanced Bayesian Network framework, designed to effectively manage the aleatoric uncertainties through the instroduction of continuous nodes, and incorporating imprecise probabilities, to handle epistemic uncertainties that arise, especially from climate change projections. This framework is applied to carbonation-induced and chloride-induced corrosion phenomena, that are the primary degradation mechanisms in concrete structures, by considering the intertwined epistemic and aleatoric uncertainties associated with environmental factors such as temperature, CO₂ concentration, and relative humidity distributions over time. The results demonstrate the significant capabilities of the enhanced Bayesian Network framework, particularly in integrating imprecise probabilities. This approach enhances structural safety analysis, offering a robust methodology for simultaneously addressing both epistemic and aleatory uncertainties, thereby advancing the assessment and mitigation of corrosion risks in concrete structures.


% By doing so, Bayesian Networks are a well-established framework to model corrosion dependencies. But due to the rigid nature and limitations in modeling aleatoric uncertaintiesof traditional Bayesian Network, it faces the challenges to account the time-dependent climate change induced impact on corrosion dynamic. To overcome these limitation we propose an enhanced Bayesian Network framework which is capable of handling continuous nodes and incorporating imprecise probabilities to tackle the epistemic uncertainties problem.
% Carbonation induced and chloride induced corrosions are considered as the 2 main degradation mechanisms of concrete structures, and are considered in the proposed framework as phenomena undergoing both epistemic and aleatoric uncertainties over some environmental variables, e.g. temperature, $CO_2$ concentration and relative humidity distributions over time. 
% Findings highlight the potentials of the enhanced Bayesian Networks with imprecise probabilities for improving the structural safety analysis providing into a robust framework for dealing with both epistemic and aleatoric uncertainties. 
% Concrete is the ultimate barrier that prevents the release of radioactive contaminants from nuclear waste repositories. Concrete suffers of two main chemical degradation processes that are expedited by climate change: carbonation-induced corrosion and chloride ingress. In this work, a Dynamic Bayesian Network (DBN) is developed for evaluating the concrete degradation induced by the change of climatic variables relevant for corrosion, e.g., temperature, relative humidity, and $CO_2$. The procedure is operationalized on a near-surface waste disposal of literature, showing that DBN can support the risk assessment of aquifer contamination due to radiological contaminants release, because enables planning the necessary remediation strategies to counteract climate change-induced concrete degradation and avoid dose intake.